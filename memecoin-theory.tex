\documentclass[a4paper,12pt]{article}
\usepackage{amsmath,amssymb}
\usepackage{geometry}
\usepackage{epigraph}
\usepackage{enumitem}
\geometry{margin=1in}

\begin{document}

\title{Signifiers of Speculation: The Semiotics of Memecoins in the Postmodern Attention Economy}
\author{Cholula}
\date{January 13th, 2025}

\maketitle

\begin{abstract}
This article delves into the intricate relationship between memecoins and their underlying cultural memes through the analytical lenses of semiotics, postmodernist philosophy, and media theory. It posits that within the memecoin ecosystem, memes initially function as the \textit{signifiers}, while the coins themselves embody the \textit{signified}, with the process of \textit{attention accumulation} operating as the mechanism of signification. However, as memecoins mature, they transcend this simple semiotic relationship to emerge as a distinct medium in their own right—one uniquely suited to securitizing and programming collective attention. Grounded in the seminal works of Jean Baudrillard, Jacques Derrida, and Marshall McLuhan, this analysis traces how memecoins evolve from mimicking traditional financial instruments to developing their native affordances as attention-based media. By examining memecoins as both cultural and technological phenomena, the study reveals deeper insights into the construction of value, the commodification of attention, and the emergence of new forms of programmable cultural exchange in digital marketplaces. It further explores how memecoins, rather than merely representing value, are evolving into infrastructure for quantifying, trading, and amplifying collective attention—suggesting a fundamental shift in how cultural and economic value are created and exchanged in the digital age.
\end{abstract}

\begin{center}
\begin{minipage}{0.9\textwidth}
\small
\begin{center}
\textbf{DISCLAIMER}
\end{center}
THIS ACADEMIC PAPER WAS GENERATED USING AN ADVANCED LARGE LANGUAGE MODEL (LLM). WHILE THE CONTENT HAS BEEN STRUCTURED TO PROVIDE COMPREHENSIVE TAXONOMICAL ANALYSIS, READERS SHOULD BE AWARE THAT LLM TECHNOLOGY, DESPITE ITS SOPHISTICATION, MAY OCCASIONALLY PRODUCE INACCURATE OR INCONSISTENT INFORMATION ("HALLUCINATIONS"). THE ANALYSIS PRESENTED HEREIN SHOULD BE CONSIDERED AS A PRELIMINARY FRAMEWORK RATHER THAN DEFINITIVE CLASSIFICATION GUIDANCE. RESEARCHERS, DEVELOPERS, AND OTHER STAKEHOLDERS ARE STRONGLY ENCOURAGED TO CONDUCT INDEPENDENT VERIFICATION OF ALL CLASSIFICATIONS AND METHODOLOGIES, SEEK QUALIFIED EXPERTISE FOR SPECIFIC IMPLEMENTATION DECISIONS, PERFORM THOROUGH DUE DILIGENCE ON ALL CATEGORICAL ASSIGNMENTS, AND STAY INFORMED ABOUT CURRENT DEVELOPMENTS IN THE DIGITAL ASSET SPACE. THE AUTHORS AND THE AI SYSTEM ASSUME NO LIABILITY FOR DECISIONS MADE BASED ON THIS CONTENT. THIS DOCUMENT IS INTENDED FOR ACADEMIC DISCUSSION PURPOSES ONLY.
\end{minipage}
\end{center}

\section{Introduction: The Postmodern Digital Economy}

\epigraph{\textit{“The spectacle is not a collection of images, but a social relation among people, mediated by images.”}}{---Guy Debord, \textit{The Society of the Spectacle}}

The advent of blockchain technology and cryptocurrencies has revolutionized conventional financial systems, engendering decentralization and democratization of economic participation. Amidst this transformation, \textit{memecoins} have emerged as a particularly perplexing and intriguing subset of cryptocurrencies. Unlike traditional cryptocurrencies that aim to solve specific technological or financial problems, memecoins derive their perceived value predominantly from internet memes—virally propagated cultural symbols and ideas that resonate within online communities. 

Memecoins such as Dogecoin, Shiba Inu, and PEPE have witnessed meteoric rises and falls in market capitalization, driven not by underlying utility or technological innovation but by the collective attention and engagement of participants within digital ecosystems. This phenomenon raises profound questions about the generation, perception, and nature of value in a postmodern context. Specifically, it invites an exploration of how cultural symbols translate into economic assets within an attention-driven economy.

This article posits that in the memecoin ecosystem, memes function as the \textit{signifiers}, transmitting meaning and cultural resonance, while the coins themselves are the \textit{signified}, representing the economic embodiment of that cultural capital. The process of \textit{attention accumulation} serves as the mechanism of signification, effectively converting the semiotic value of memes into quantifiable economic value through networked participation and speculation.

By leveraging the theoretical frameworks of semiotics and postmodernist philosophy, particularly the works of Ferdinand de Saussure, Jean Baudrillard, Jacques Derrida, and Gilles Deleuze and Félix Guattari, this study aims to dissect the symbiotic relationship between memes and memecoins. It explores how this relationship destabilizes traditional notions of value and representation, challenging established economic paradigms and offering insights into the postmodern condition as manifested in digital economies.

The analysis unfolds through a comprehensive examination of the semiotic structures underpinning memecoins, an exploration of attention as an economic commodity, and case studies illustrating these dynamics in practice. The study further delves into the ethical implications and the broader socio-economic impact of memecoins, ultimately positioning them as emblematic of the postmodern spectacle within the digital marketplace.

\section{Theoretical Framework: Postmodern Semiotics}

\subsection{Foundations of Semiotics: Saussurean Linguistics}

Semiotics, the study of signs and symbols as elements of communicative behavior, provides a critical framework for understanding how meaning is constructed and conveyed. Ferdinand de Saussure's structural linguistics lays the groundwork with his dyadic model of the \textit{sign}, comprising the \textit{signifier} and the \textit{signified}~\cite{saussure1986course}. The \textit{signifier} refers to the form that the sign takes, while the \textit{signified} denotes the concept it represents. This relationship is arbitrary yet socially agreed upon within a linguistic community.

In applying Saussure's model to memecoins, the internet meme operates as the signifier—a tangible or perceivable form such as an image, phrase, or video that carries cultural significance. The memecoin represents the signified, embodying the economic concept or value that investors and participants exchange and speculate upon. The arbitrariness of this association is evident; there is no intrinsic reason why a particular meme should be linked to a cryptocurrency, yet the collective acceptance within the digital community establishes and reinforces this connection.

\subsection{Baudrillard's Simulacra and Hyperreality}

Jean Baudrillard's exploration of \textit{simulacra} and \textit{simulation} extends the discussion of signs into the realm of postmodernism. Baudrillard posits that in contemporary society, signs and symbols have proliferated to the point where they no longer refer to an underlying reality but instead to other signs, creating a state of hyperreality~\cite{baudrillard1994simulacra}. In hyperreality, the distinction between the real and the simulated collapses, and the simulation becomes more real than reality itself.

Memecoins epitomize this concept of hyperreality. They are financial instruments devoid of traditional economic underpinnings such as tangible assets or services. Instead, their value is constructed through the perpetual circulation and referencing of memes, which themselves are layers of simulation detached from any original referent. The memecoin thus becomes a simulacrum—a copy without an original—that derives its perceived value from a hyperreal system of signs perpetuated by collective participation and media representation.

\subsection{Derrida's \textit{Différance} and Deferred Meaning}

Jacques Derrida's concept of \textit{différance} introduces the idea that meaning is always deferred, never fully present or fixed~\cite{derrida1976grammatology}. Language, and by extension any system of signs, functions through a play of differences where each sign gains meaning in relation to others. This perpetual deferral implies that any attempt to pin down a definitive meaning is inherently unstable.

Applied to memecoins, the meaning and value of a meme—and consequently the associated coin—are inherently fluid and subject to continual reinterpretation within the digital discourse. The meme's resonance can shift rapidly based on cultural trends, social sentiments, and viral dynamics. This deferral of meaning contributes to the volatility of memecoin valuations, as the economic value is contingent upon the transient and fluctuating cultural significance of the meme.

\subsection{Deleuze and Guattari's Rhizomatic Structures}

Gilles Deleuze and Félix Guattari introduce the concept of the \textit{rhizome} as a model for understanding knowledge and societal structures that are non-hierarchical, interconnected, and multi-centered~\cite{deleuze1987thousand}. The rhizomatic structure resists centralized control and linear progression, instead favoring a networked approach where any point can connect to any other.

The proliferation of memes and memecoins can be conceptualized as a rhizomatic network, where ideas and value propagate through decentralized and unpredictable pathways. This aligns with the decentralized nature of blockchain technology and the way memes spread virally across platforms, unconstrained by traditional gatekeepers of information or capital.

\subsection{Mythologies and Cultural Significance}

Roland Barthes, in \textit{Mythologies}, explores how everyday objects and cultural artifacts can carry ideological meanings and become mythic~\cite{barthes1972mythologies}. Memes, as modern cultural artifacts, function similarly by encapsulating and disseminating ideas, values, and critiques within society.

Memecoins can thus be seen as the commodification of these modern myths. They transform cultural symbols into economic entities, enabling participants to invest in and speculate upon the collective myths that permeate digital culture. This process highlights the intersection of cultural semiotics and economic behavior in the postmodern digital economy.

\section{Attention as the Signifying Process}

\subsection{The Attention Economy: Commodification of Focus}

The concept of the attention economy posits that in an environment saturated with information, attention becomes a scarce and valuable resource~\cite{goldhaber1997attention}. Michael H. Goldhaber argues that the allocation of attention is a form of economic exchange, where individuals and entities vie for the limited focus of others.

In the context of memecoins, attention functions as both currency and capital. Memes capture the attention of users through humor, relatability, or novelty, which in turn amplifies their visibility and cultural salience. This heightened attention increases the perceived value of the associated memecoin, as more individuals become aware of and interested in it. The memecoin's market performance is thus directly correlated with the magnitude and intensity of collective attention it garners.

\subsection{Deleuze and Guattari's Desiring-Production}

In \textit{Anti-Oedipus: Capitalism and Schizophrenia}, Deleuze and Guattari introduce the notion of \textit{desiring-production}, where desire is a productive force that underlies all social and economic systems~\cite{deleuze1983anti}. They conceptualize individuals as \textit{desiring-machines} that are constantly engaged in processes of production and consumption driven by desire.

Attention accumulation within digital communities can be framed as an instance of desiring-production. Users desire engagement, recognition, and participation in cultural phenomena. Memes act as catalysts for this desire, initiating cycles of production (creating and sharing content) and consumption (viewing and interacting with content). Memecoins capitalize on this dynamic by providing a means to monetize the collective desire embedded within memes, transforming cultural participation into economic transactions.

\subsection{Network Effects and Viral Dynamics}

Network effects describe the phenomenon where the value of a product or service increases as more people use it~\cite{shapiro1999information}. In digital platforms, viral dynamics amplify network effects, enabling rapid dissemination of content through social connections.

Memecoins benefit from network effects as the viral spread of memes draws in new participants, investors, and speculators. Each additional user not only contributes to the pool of potential buyers but also enhances the visibility and legitimacy of the memecoin. The interdependence between meme virality and memecoin value exemplifies how network effects drive economic outcomes in the attention economy.

\subsection{Algorithmic Amplification and Platform Mediation}

Social media platforms utilize algorithms to curate and surface content, often prioritizing engagement metrics such as likes, shares, and comments~\cite{gillespie2014algorithm}. This algorithmic amplification can significantly impact the reach and influence of memes.

The mediation of attention through platform algorithms introduces a layer of complexity in the signifying process. The visibility of memes—and thus the potential value of associated memecoins—is partially determined by opaque algorithmic processes that favor certain types of content. This adds an element of unpredictability and reinforces the postmodern notion of decentralized and fragmented meaning-making.

\section{Case Studies: Memecoins as Cultural Phenomena}

\subsection{Dogecoin: From Parody to Phenomenon}

\subsubsection{Origins and Cultural Context}

Dogecoin was created in 2013 by software engineers Billy Markus and Jackson Palmer as a satirical response to the proliferation of alternative cryptocurrencies~\cite{higgins2018dogecoin}. It features the Shiba Inu dog from the "Doge" meme, which was popular on the internet at the time for its humorous portrayal of an internal monologue expressed in broken English and Comic Sans font.

The choice of the "Doge" meme as the emblem for the cryptocurrency was intentionally playful, subverting the seriousness typically associated with financial instruments. This aligns with Baudrillard's concept of hyperreality, as the coin blurs the boundaries between reality and parody, becoming a self-referential sign that derives meaning from its own circulation.

\subsubsection{Community Formation and Charitable Actions}

Dogecoin quickly developed a dedicated online community that embraced the meme culture and leveraged the cryptocurrency for philanthropic endeavors, such as sponsoring NASCAR drivers and funding clean water projects~\cite{ryan2018dogecoin}. These collective actions reinforced the cultural significance of Dogecoin, transforming it from a mere parody into a vehicle for communal identity and social impact.

The communal engagement exemplifies Deleuze and Guattari's desiring-production, where collective desire produces tangible outcomes beyond mere financial speculation. Attention accumulation fueled by shared values and humor translated into both economic value and social capital.

\subsubsection{Market Volatility and Celebrity Endorsements}

The volatility of Dogecoin's market value has been influenced significantly by attention spikes resulting from celebrity endorsements and social media trends. High-profile figures like Elon Musk have tweeted about Dogecoin, causing rapid fluctuations in its price~\cite{pan2021elon}.

This phenomenon underscores the role of attention as a signifying process. The memecoin's value becomes a reflection of the collective focus and sentiment at any given moment, with meanings and valuations deferred and reshaped by external influences, resonating with Derrida's concept of \textit{différance}.

\subsection{Shiba Inu and PEPE: The Replication of Memetic Value}

\subsubsection{Shiba Inu: The "Dogecoin Killer"}

Shiba Inu was launched in 2020 as an Ethereum-based token, branding itself as the "Dogecoin Killer"~\cite{buterin2021shiba}. It capitalizes on the same meme imagery as Dogecoin but introduces decentralized finance (DeFi) features such as a decentralized exchange and staking mechanisms.

The replication and iteration of the Shiba Inu meme signify the rhizomatic spread of cultural symbols in the digital space. The memecoin's creators and community leverage the existing cultural resonance to attract attention, but also introduce new dimensions, such as DeFi functionalities, to differentiate and add layers of meaning. This continual evolution highlights the fluidity and deferral of meaning in the memecoin space.

\subsubsection{PEPE: Embracing Controversial Memes}

PEPE coin associates itself with the "Pepe the Frog" meme, a symbol that has undergone significant transformation in cultural perception~\cite{allen2017pepe}. Originally a benign comic character, Pepe became co-opted by various online subcultures, sometimes associated with controversial or extremist ideologies.

The adoption of Pepe as a memecoin symbol brings complex semiotic layers, involving issues of authorship, appropriation, and ideological connotations. Investing in PEPE coin implicates participants in the cultural narratives attached to the meme, further complicating the signifier-signified relationship. This case exemplifies how memecoins can serve as sites of cultural contestation and reflection.

\subsection{Memecoins as Instruments of Speculation and Social Commentary}

The rapid emergence and fluctuation of memecoins also function as a critique of traditional financial systems. They expose the vulnerabilities and absurdities within markets that can be swayed by intangibles like memes and social media trends.

This dynamic resonates with Baudrillard's notion of the spectacle, where the representation overshadows substance, and illusions are traded as reality. Memecoins become instruments through which participants can engage in speculative play, challenging conventional notions of value, and highlighting the performative aspects of economic transactions.

\section{Implications: The Semiotics of Value in the Attention Economy}

\subsection{Destabilizing Traditional Economic Models}

The memecoin phenomenon destabilizes classical economic theories that ground value in labor, utility, or scarcity~\cite{smith1776wealth}. Value in the memecoin context is decoupled from these traditional indicators and is instead constructed through cultural engagement and attention metrics.

This shift aligns with Marx's critique of commodity fetishism, where social relationships are mediated through objects whose value is perceived as inherent rather than socially constructed~\cite{marx1867capital}. Memecoins invert this by making the social construction of value explicit, foregrounding the role of collective perception and cultural resonance.

\subsection{The Commodification of Culture and Attention}

Memecoins exemplify the commodification of culture, where memes—once ephemeral expressions of collective creativity—are transformed into financial assets. This raises questions about the appropriation and monetization of cultural symbols, often without consent or compensation to original creators.

The attention economy, as facilitated by platform capitalism, turns user engagement and cultural production into profitable data streams for corporations~\cite{srnicek2017platform}. Memecoins extend this commodification by enabling direct speculation on cultural artifacts, further entrenching the marketization of social and cultural life.

\subsection{Ethical Considerations and Potential for Exploitation}

The speculative nature of memecoins presents ethical concerns related to financial risk and misinformation. Rapid value fluctuations can lead to significant financial losses for uninformed investors, often exacerbated by misleading marketing and hype~\cite{gudgeon2020defi}.

Moreover, the manipulation of attention through coordinated campaigns or algorithmic targeting raises issues of fairness and transparency. The asymmetry of information and power between creators, influencers, and everyday participants can result in exploitative dynamics.

\subsection{Regulatory Challenges and Responses}

The decentralized and transnational nature of memecoins poses challenges for regulatory frameworks designed for traditional financial instruments~\cite{fenwick2018regulating}. Authorities grapple with balancing innovation and consumer protection, as well as addressing issues of money laundering and fraud.

Regulatory responses may also impact the dynamics of attention and value in the memecoin ecosystem, as legal interventions can alter perceptions of legitimacy and risk, influencing participant behavior.

\section{Memecoins as New Medium: Beyond Financial Mimesis}

\epigraph{\textit{"We shape our tools and thereafter our tools shape us."}}{---Marshall McLuhan}

\subsection{The Evolution of Media Forms}

Marshall McLuhan's assertion that "the content of a new medium is always an old medium"~\cite{mcluhan1964understanding} provides a crucial framework for understanding memecoins' evolutionary trajectory. Just as television initially presented radio with pictures, and personal computers first mimicked typewriters, memecoins began by imitating traditional financial instruments while incorporating memetic elements.

This pattern aligns with what McLuhan termed the "rear-view mirror" effect—our tendency to understand new media through the lens of familiar forms~\cite{mcluhan1967medium}. However, as Bolter and Grusin note in their theory of remediation, new media forms eventually transcend mere imitation to develop their own unique affordances and cultural significance~\cite{bolter1999remediation}.

\subsection{From Financial Mimesis to Attention Infrastructure}

\subsubsection{Initial Stage: Cryptocurrency Mimesis}

In their initial incarnation, memecoins primarily mimicked existing cryptocurrency patterns, operating as speculative tokens enhanced by memetic content. This reflects what Friedrich Kittler describes as the "discourse network"—the ways new media initially conform to existing systems of inscription and transmission~\cite{kittler1990discourse}.

\subsubsection{Emergent Properties: Attention as Medium}

As memecoins mature, they increasingly manifest what McLuhan would term their own "message"—not their content (the memes themselves) but their broader social and cultural implications. The true "message" of memecoins lies in their capacity to transform attention into a programmable, tradeable asset class.

This transformation echoes Walter Benjamin's observations about mechanical reproduction and art's loss of "aura"~\cite{benjamin1935work}. Just as mechanical reproduction democratized access to art, memecoins democratize the ability to capture and monetize collective attention, creating what could be termed an "attention aura" that can be quantified and traded.

\subsection{Native Capabilities: The Medium's True Message}

\subsubsection{Programmable Attention Markets}

Drawing on Vilém Flusser's concept of "technical images"~\cite{flusser2011into}, memecoins can be understood as "technical attention"—attention that has been abstracted into programmable units. This enables:

\begin{itemize}
    \item Automated market-making for attention-based assets
    \item Algorithmic distribution of engagement rewards
    \item Quantifiable metrics for cultural impact
\end{itemize}

\subsubsection{Network-Effect Securitization}

Manuel Castells' theory of the network society~\cite{castells2000rise} helps explain how memecoins function as instruments for securitizing network effects:

\begin{itemize}
    \item Communities can capture value from their collective attention
    \item Network growth becomes directly linked to token value
    \item Cultural capital gains immediate financial expression
\end{itemize}

\subsection{The Attention Economy's Native Medium}

Michael H. Goldhaber's prescient analysis of the attention economy~\cite{goldhaber1997attention} gains new relevance when applied to memecoins. If attention is indeed the dominant currency of the digital age, then memecoins represent its native financial instrument—a medium specifically evolved to capture, trade, and amplify attention flows.

This evolution suggests what Katherine Hayles terms "technogenesis"~\cite{hayles2012technogenesis}—the co-evolution of humans and technology. As memecoins mature, they're likely to develop increasingly sophisticated mechanisms for:

\begin{itemize}
    \item Quantifying cultural resonance
    \item Automating viral distribution
    \item Incentivizing creative participation
    \item Governing attention-based communities
\end{itemize}

\subsection{Future Implications: Beyond Financial Instruments}

The emergence of memecoins as attention's native medium suggests a fundamental shift in how cultural value is created and exchanged. As Friedrich Kittler argues, media determine our situation~\cite{kittler1999gramophone}. In this light, memecoins may represent not just a new financial instrument but a new grammar for cultural exchange—one where attention, financial value, and cultural significance become programmatically intertwined.

This transformation echoes McLuhan's tetrad of media effects~\cite{mcluhan1988laws}:

\begin{itemize}
    \item Enhancement: Amplifies community-driven value creation
    \item Obsolescence: Makes traditional attention metrics obsolete
    \item Retrieval: Brings back direct community ownership of cultural assets
    \item Reversal: Transforms passive consumption into active participation
\end{itemize}

\begin{thebibliography}{99}

\bibitem{saussure1986course}
F.~de~Saussure,
  \emph{Course in General Linguistics}.
  Open Court Publishing, 1986.

\bibitem{baudrillard1994simulacra}
J.~Baudrillard,
  \emph{Simulacra and Simulation}.
  University of Michigan Press, 1994.

\bibitem{derrida1976grammatology}
J.~Derrida,
  \emph{Of Grammatology}.
  Johns Hopkins University Press, 1976.

\bibitem{deleuze1983anti}
G.~Deleuze and F.~Guattari,
  \emph{Anti-Oedipus: Capitalism and Schizophrenia}.
  University of Minnesota Press, 1983.

\bibitem{deleuze1987thousand}
G.~Deleuze and F.~Guattari,
  \emph{A Thousand Plateaus: Capitalism and Schizophrenia}.
  University of Minnesota Press, 1987.

\bibitem{barthes1972mythologies}
R.~Barthes,
  \emph{Mythologies}.
  Farrar, Straus and Giroux, 1972.

\bibitem{goldhaber1997attention}
M.~H.~Goldhaber,
  ``The Attention Economy and the Net,''
  \emph{First Monday}, vol.~2, no.~4, 1997.

\bibitem{gillespie2014algorithm}
T.~Gillespie,
  ``The Relevance of Algorithms,''
  \emph{Media Technologies: Essays on Communication, Materiality, and Society}, pp. 167--194, 2014.

\bibitem{higgins2018dogecoin}
S.~Higgins,
  ``A Brief History of Dogecoin, the Internet's Favourite Cryptocurrency,''
  \emph{CoinDesk}, 2018. [Online]. Available: \url{https://www.coindesk.com/markets/2018/01/09/a-brief-history-of-dogecoin-the-internets-favourite-cryptocurrency/}

\bibitem{ryan2018dogecoin}
F.~Ryan,
  ``How Dogecoin Became the Internet's Favourite Cryptocurrency,''
  \emph{The Guardian}, 2018. [Online]. Available: \url{https://www.theguardian.com/technology/2018/mar/01/dogecoin-internet-favourite-cryptocurrency-shiba-inu}

\bibitem{pan2021elon}
D.~Pan,
  ``Elon Musk's Tweets Send Dogecoin to a New Record High,''
  \emph{Bloomberg}, 2021. [Online]. Available:
  \url{https://www.bloomberg.com/news/articles/2021-05-04/dogecoin-soars-to-record-after-elon-musk-twitter-exchange}

\bibitem{buterin2021shiba}
V.~Buterin,
  ``Vitalik Buterin Burns 410 Trillion Shiba Inu Tokens,''
  \emph{Ethereum Foundation}, 2021. [Online]. Available:
  \url{https://ethereum.org/en/blog/vitalik-burns-shiba-inu-tokens/}

\bibitem{allen2017pepe}
M.~Allen,
  ``How Pepe the Frog Became a Nazi Trump Supporter and Alt-Right Symbol,''
  \emph{The Daily Beast}, 2017. [Online]. Available:
  \url{https://www.thedailybeast.com/how-pepe-the-frog-became-a-nazi-trump-supporter-and-alt-right-symbol}

\bibitem{shapiro1999information}
C.~Shapiro and H.~Varian,
  \emph{Information Rules: A Strategic Guide to the Network Economy}.
  Harvard Business School Press, 1999.

\bibitem{smith1776wealth}
A.~Smith,
  \emph{An Inquiry into the Nature and Causes of the Wealth of Nations}.
  W. Strahan and T. Cadell, 1776.

\bibitem{marx1867capital}
K.~Marx,
  \emph{Capital: A Critique of Political Economy}, vol.~1.
  Penguin Books, 1990.

\bibitem{srnicek2017platform}
N.~Srnicek,
  \emph{Platform Capitalism}.
  Polity Press, 2017.

\bibitem{gudgeon2020defi}
L.~Gudgeon, et al.,
  ``DeFi Protocol Risks: the Paradox of DeFi,''
  \emph{arXiv preprint arXiv:2009.14021}, 2020.

\bibitem{fenwick2018regulating}
M.~Fenwick, J.~McCahery, and E.~Vermeulen,
  ``Regulating a Revolution: From Regulatory Sandboxes to Smart Regulation,''
  \emph{Fordham Journal of Corporate \& Financial Law}, vol.~23, no.~1, pp. 31--103, 2018.

\end{thebibliography}

\end{document}
