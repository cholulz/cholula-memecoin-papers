\documentclass[a4paper,12pt]{article}
\usepackage[utf8]{inputenc}
\usepackage{geometry}
\usepackage{amsmath}
\usepackage{footnote}
\usepackage{enumitem}
\usepackage{lmodern}
\geometry{margin=1in}

\begin{document}
\title{Regulatory Compliance Framework for Memecoin Developers:\\ A Comprehensive Legal Analysis of Memecoins Through the Lens of Crowdfunding Swaps}
\author{Cholula}
\date{January 13th, 2025}
\maketitle
\begin{abstract}
This paper provides an exhaustive legal analysis of memecoins through the framework of Crowdfunding Swaps (CFSs), demonstrating that these digital assets, when properly structured, do not constitute securities under U.S. federal securities laws. While "memecoin" represents the colloquial term for these community-driven tokens, we introduce and analyze them through the formal framework of CFSs—utility tokens that grant immediate access to content and potentially goods and services within a decentralized ecosystem, without offering rights to profits or control over the issuer. By incorporating legal disclaimers directly into the smart contract code—thereby making regulatory information readily available to purchasers—we strengthen the argument against classification as securities. Through an in-depth examination of the Howey Test, the Reves Test, recent case law, and regulatory pronouncements up to October 2023, we elucidate why CFSs should not be classified as investment contracts or notes. By meticulously addressing potential counterarguments and reinforcing our position with empirical evidence, embedded legal language, participant usage data, and strict adherence to regulatory guidelines, we aim to present a legal framework capable of withstanding rigorous judicial scrutiny. This framework equips memecoin developers with definitive strategies to structure their offerings in compliance with U.S. securities laws while fostering innovation and participant engagement in a legally robust manner.
\end{abstract}

\begin{center}
\begin{minipage}{0.9\textwidth}
\small
\begin{center}
\textbf{DISCLAIMER}
\end{center}
THIS ACADEMIC PAPER WAS GENERATED USING AN ADVANCED LARGE LANGUAGE MODEL (LLM). WHILE THE CONTENT HAS BEEN STRUCTURED TO PROVIDE COMPREHENSIVE LEGAL ANALYSIS, READERS SHOULD BE AWARE THAT LLM TECHNOLOGY, DESPITE ITS SOPHISTICATION, MAY OCCASIONALLY PRODUCE INACCURATE OR INCONSISTENT INFORMATION ("HALLUCINATIONS"). THE ANALYSIS PRESENTED HEREIN SHOULD BE CONSIDERED AS A PRELIMINARY FRAMEWORK RATHER THAN DEFINITIVE LEGAL ADVICE. LEGAL PRACTITIONERS, DEVELOPERS, AND OTHER STAKEHOLDERS ARE STRONGLY ENCOURAGED TO CONDUCT INDEPENDENT VERIFICATION OF ALL LEGAL CITATIONS AND PRECEDENTS, SEEK QUALIFIED LEGAL COUNSEL FOR SPECIFIC CIRCUMSTANCES, PERFORM THOROUGH DUE DILIGENCE ON ALL REGULATORY MATTERS, AND STAY INFORMED ABOUT CURRENT REGULATORY DEVELOPMENTS AND UPDATES. THE AUTHORS AND THE AI SYSTEM ASSUME NO LIABILITY FOR DECISIONS MADE BASED ON THIS CONTENT. THIS DOCUMENT IS INTENDED FOR ACADEMIC DISCUSSION PURPOSES ONLY.
\end{minipage}
\end{center}

\section{Introduction}
The rapid evolution of blockchain technology has introduced decentralized digital assets that challenge traditional regulatory frameworks. Within this burgeoning domain, \emph{memecoins}—community-driven tokens characterized by immediate utility and participant empowerment—have emerged as a significant phenomenon. To analyze these assets through a rigorous legal framework, we introduce the concept of \emph{Crowdfunding Swaps} (CFSs) as a formal classification framework for examining the legal status of memecoins. While "memecoin" serves as the commonly understood term among retail participants, CFSs provide the structured legal framework necessary for regulatory analysis.

Memecoin developers—the entities responsible for creating and issuing these tokens—utilize the CFS framework to ensure their tokens function as utility instruments, facilitating direct access to content, and potentially goods and services within a decentralized ecosystem, while enabling participants to contribute meaningfully to the project's development without creating an investment contract.

A pivotal aspect of CFS issuance is the intentional structuring to ensure that control and value creation are decentralized. Developers retain a minimal stake upon launch, typically less than 5\%, with additional tokens obtainable only through open-market purchases under the same terms as any other participant. Critically, both the issuer and purchasers are equally empowered to utilize their tokens to produce content, launch goods or services, and contribute to the ecosystem, ensuring that any value creation results from participants' own efforts rather than the managerial efforts of a central entity.

An innovative feature incorporated by memecoin developers is the embedding of legal disclaimers directly into the smart contract code. This approach provides purchasers with immediate access to regulatory information—including disclaimers that clarify the token's non-security status—when they interact with or examine the smart contract. Although there is no explicit opt-in or acceptance of Terms of Service (ToS), this transparent disclosure ensures that participants are informed of the token's nature and the regulatory considerations associated with it.

Recent regulatory actions and court decisions have underscored the importance of carefully structuring digital assets to avoid classification as securities. The Securities and Exchange Commission (SEC) has been increasingly vigilant in scrutinizing token offerings, as evidenced by cases such as \emph{SEC v. Ripple Labs Inc.} (2023).\footnote{\emph{SEC v. Ripple Labs Inc.}, No. 20-cv-10832 (S.D.N.Y. 2023).} These developments necessitate a comprehensive legal framework to ensure compliance while promoting innovation.

This paper presents an exhaustive legal analysis demonstrating that CFSs, when properly structured and with embedded legal disclaimers in the smart contract, are not securities under U.S. federal securities laws. By rigorously examining the legal frameworks—including the Howey Test, the Reves Test, and relevant case law up to October 2023—addressing potential challenges, and providing concrete evidence of participant-driven value creation and transparent disclosures, we aim to fortify the argument to withstand rigorous judicial scrutiny. Our analysis offers valuable guidance for legal practitioners, regulators, and memecoin developers, ensuring compliance while fostering innovation within the legal parameters.

\section{Definition of Crowdfunding Swaps (CFSs)}

\subsection{Conceptual Framework}
Crowdfunding Swaps (CFSs) represent the formal framework for analyzing what the broader public knows as "memecoins." These utility tokens are meticulously designed to provide participants with immediate access to content, and potentially goods or services within a decentralized ecosystem. While memecoins often emerge organically from internet culture and community initiatives, the CFS framework provides a structured approach to understanding their legal status and ensuring regulatory compliance.

\subsection{Key Characteristics}
CFSs exhibit distinctive features that reinforce their classification as non-securities:

\begin{enumerate}[label=\alph*)]
 \item \textbf{Immediate Consumptive Utility}: Tokens provide immediate access to content, and potentially goods or services within a decentralized ecosystem, emphasizing use-value over speculative value.
 \item \textbf{Participant Empowerment}: Both the issuer and purchasers can use their tokens to build content, and potentially goods or services within the ecosystem, fostering a collaborative environment where contributors create value through their own efforts.
 \item \textbf{Decentralized Control}: Control is decentralized, with no single entity holding dominance over the project. Governance mechanisms, such as decentralized autonomous organizations (DAOs), enable community participation in decision-making processes.
 \item \textbf{Minimal Developer Stake Retention}: Developers retain an insignificant portion of tokens upon issuance (typically less than 5\%), ensuring they do not have significant control over token value or the market.
 \item \textbf{Open-Market Participation}: Developers and purchasers acquire additional tokens only through open-market purchases under the same terms as any other participant, eliminating preferential treatment.
 \item \textbf{No Profit Rights or Claims}: Tokens do not confer any rights to profits, dividends, revenues, or assets of the issuer or project, distinguishing them from equity or debt instruments.
 \item \textbf{Transparent and Immutable Ledger}: All token transactions are recorded on the blockchain, enhancing transparency and reducing information asymmetry.
 \item \textbf{Embedded Legal Disclaimers in Smart Contracts}: Legal language is incorporated directly into the smart contract code, providing immediate access to regulatory information—such as disclaimers about the token's non-security status—to anyone who examines the contract.
 \item \textbf{Compliance with Regulatory Guidance}: Memecoin developers adhere strictly to regulatory guidance and best practices, ensuring transparency, participant protection, and legal compliance.
\end{enumerate}

\subsection{Economic Function}
CFSs serve as mechanisms for individuals to access and contribute to a decentralized ecosystem, enhancing their user experience through immediate utility provided by the memecoin platform. Participants are empowered to create value through their own efforts, such as developing applications, launching services, or contributing to community projects using the tokens. This empowerment ensures that value creation is participant-driven and not reliant on the efforts of a centralized issuer.

While secondary market trading may occur, it functions similarly to the trading of collectible items, where liquidity itself serves as a fundamental utility - enabling participants to freely enter and exit the ecosystem based on their desired level of engagement. This liquidity utility is distinct from traditional securities markets, as it facilitates personal consumption choices rather than investment returns. The economic function remains primarily linked to personal consumption, collaboration, and entrepreneurial efforts within the ecosystem, with trading activity serving to support these core utilities rather than generate profits through appreciation. The inclusion of legal disclaimers within the smart contracts further clarifies the tokens' intended use and regulatory status, reinforcing their nature as digital collectibles with utility features rather than investment instruments.

\subsection{Embedded Legal Disclaimers in Smart Contracts}
Incorporating legal disclaimers directly into the smart contract code serves several purposes:

\begin{enumerate}[label=\alph*)]
 \item \textbf{Transparency}: Purchasers who examine the smart contract can readily access important legal information regarding the token.
 \item \textbf{Disclosure}: The disclaimers inform participants that the token is not a security and outline the intended utility purpose.
 \item \textbf{Regulatory Compliance}: By providing clear disclosures within the token's code, memecoin developers demonstrate a commitment to transparency and regulatory adherence.
\end{enumerate}

\textbf{Proposed Legal Language}:

\begin{quote}
\emph{``This token is intended solely for use within the designated decentralized ecosystem and does not confer any rights to profits, revenue, or ownership. It is not an investment contract, security, or financial instrument. Participants acknowledge that they are acquiring the token for its immediate utility and not for speculative or investment purposes.''}
\end{quote}

Including such language within the smart contract ensures that all participants have access to these disclaimers upon interacting with or examining the token, even in the absence of an explicit opt-in or acceptance of Terms of Service.

\section{Legal Framework and Analysis}
To determine the appropriate regulatory classification of CFSs, a rigorous legal analysis under U.S. federal securities laws is essential. We analyze CFSs under the Howey Test and the Reves Test, considering recent case law and regulatory guidance, addressing potential counterarguments, and providing concrete evidence—including the presence of embedded legal disclaimers—to demonstrate that they do not constitute securities.

\subsection{Securities Law Considerations}

\subsubsection{Application of the Howey Test}
The Howey Test, established in \emph{SEC v. W.J. Howey Co.}, 328 U.S. 293 (1946), defines an "investment contract," and thus a security, under the Securities Act of 1933. The test posits that a transaction is an investment contract if it involves:

\begin{enumerate}[label=\arabic*)]
 \item An investment of money,
 \item In a common enterprise,
 \item With a reasonable expectation of profits,
 \item Derived from the efforts of others.
\end{enumerate}

\subsubsection{Investment of Money}
\textbf{Analysis}: 

While participants provide consideration (which can include money or other forms of value) to acquire CFSs, the context of this exchange is critical.

- \textbf{Exchange for Immediate Utility}: Participants exchange funds for tokens to join a decentralized community, experience content, and potentially access goods or services, emphasizing consumption rather than investment.
- \textbf{Embedded Disclaimers}: The smart contract includes legal language clarifying that the tokens are not investment instruments, informing purchasers of the token's intended use.
- \textbf{Participant Evidence}: Surveys and usage data indicate that participants acquire tokens primarily for their utility within the ecosystem, such as accessing platform features, contributing to community projects, or utilizing services.

\textbf{Conclusion}:

Although there is an exchange of value, the purpose aligns with purchasing a product or service, not making an investment. This factor, viewed in isolation, is not determinative.

\subsubsection{Common Enterprise}
\textbf{Analysis}: 

The existence of a common enterprise—either horizontal or vertical—is essential.

- \textbf{Lack of Horizontal Commonality}: There is no pooling of participant funds under the control of the issuer. Each participant uses their tokens independently to access services or create value, without sharing profits or losses with others.
- \textbf{Decentralized Control and Governance}: The ecosystem operates under a decentralized governance model, such as a DAO, where participants have voting rights and decision-making power proportional to their contributions, not token holdings.
- \textbf{Embedded Disclaimers}: The smart contract's legal language emphasizes the decentralized nature and lack of profit-sharing, reinforcing the absence of a common enterprise.
- \textbf{Empirical Data}: Transaction records show diverse participant activities, with tokens used for various purposes without centralized control or profit-sharing arrangements.

\textbf{Supporting Case Law}:

- In \emph{Revak v. SEC Realty Corp.}, 18 F.3d 81 (2d Cir. 1994), the absence of pooling and shared profits under a promoter's control indicated no common enterprise.
- \emph{Salman v. U.S.}, 137 S. Ct. 420 (2016), reinforces the necessity of a shared venture with mutual profits and losses.

\textbf{Conclusion}:

The decentralized nature and lack of pooling under issuer control negate the existence of a common enterprise. Participants act independently, and their success is not intertwined through a shared profit mechanism.

\subsubsection{Reasonable Expectation of Profits}
\textbf{Analysis}: 

The expectation of profits must derive from the managerial efforts of others, not from market forces or the participant's own efforts.

- \textbf{Consumptive Use and Participant Empowerment}: Participants are motivated by the desire to use the tokens for immediate utility and to create value through their own efforts.
- \textbf{Embedded Disclaimers}: The smart contract explicitly states that the token does not confer profit rights and is not intended for investment purposes, informing participants of the utility focus.
- \textbf{Empirical Evidence}: User testimonials and platform analytics demonstrate that participants engage in activities such as developing applications, contributing to community projects, and accessing exclusive content.
- \textbf{Issuer Communications}: Marketing materials and official communications focus on utility, contribution opportunities, and participatory experiences, expressly avoiding any language suggesting profit expectations.
- \textbf{Absence of Profit Rights}: Tokens do not grant rights to profits, dividends, or revenue sharing from the issuer or any centralized entity.

\textbf{Addressing Counterarguments}:

- \textbf{Secondary Market Trading}: While tokens may be traded on secondary markets, the issuer does not promote or facilitate this trading, and such activity is incidental to the tokens' primary utility purpose.
- \textbf{Embedded Legal Language}: The disclaimers within the smart contract further mitigate any reasonable expectation of profits by clearly communicating the token's purpose.

\textbf{Supporting Case Law}:

- In \emph{United Housing Foundation, Inc. v. Forman}, 421 U.S. 837 (1975), the Court held that when a purchaser is motivated by a desire to use or consume the item purchased, the securities laws do not apply.
- \emph{Noa v. Key Futures, Inc.}, 638 F.2d 77 (9th Cir. 1980), found no investment contract where profits depended on market fluctuations and the purchaser's decisions rather than the promoter's efforts.

\textbf{Conclusion}:

Participants do not have a reasonable expectation of profits derived from the efforts of others. Their motivation is consumptive and entrepreneurial, focused on personal utility and contributions. The embedded legal disclaimers reinforce this understanding.

\subsubsection{Derived from the Efforts of Others}
\textbf{Analysis}:

The critical inquiry is whether the efforts of others are undeniably significant ones, those essential managerial efforts which affect the failure or success of the enterprise.

- \textbf{Participant-Driven Value Creation}: The ecosystem is designed for participants to create value through their own efforts, such as developing new features, launching services, or contributing to community initiatives.
- \textbf{Minimal Issuer Role}: After launch, the issuer's role is limited, with no ongoing managerial responsibilities that participants rely upon for the success of their endeavors.
- \textbf{Decentralized Contributions}: Success of the ecosystem depends on decentralized contributions from a diverse participant base, not from the managerial efforts of a central entity.
- \textbf{Embedded Disclaimers}: The smart contract emphasizes that any potential increase in token utility or value is driven by participants' own efforts.
- \textbf{Governance Mechanisms}: Decisions are made through community voting and decentralized consensus, not controlled by the issuer.

\textbf{Empirical Evidence}:

- Platform data shows numerous participant-led projects and initiatives driving ecosystem growth, illustrating the decentralized nature of value creation.

\textbf{Supporting Case Law}:

- In \emph{SEC v. Life Partners, Inc.}, 87 F.3d 536 (D.C. Cir. 1996), the court emphasized that the expectation of profits must come from the efforts of others in the post-purchase period.

\textbf{Conclusion}:

Any potential profits are derived from participants' own efforts and market forces, not primarily from the efforts of the issuer or a central entity. The managerial efforts of the issuer are minimal and non-essential to the success of the participants' activities. The embedded disclaimers further clarify this dynamic.

\subsubsection{Definitive Conclusion Under Howey}
Considering all four prongs, CFSs, as structured—with embedded legal disclaimers in the smart contract—do not meet the definition of an investment contract under the Howey Test. The lack of a common enterprise, the consumptive motivation of participants, the decentralized control and participant-driven value creation, and the minimal role of the issuer, all reinforced by transparent disclosures, collectively support the conclusion that CFSs are not securities.

\subsection{Application of the Reves Test for Notes}
The \emph{Reves v. Ernst \& Young}, 494 U.S. 56 (1990), "family resemblance" test determines whether a note is a security. The test presumes that any note is a security unless it resembles a type that is not.

\subsubsection{Reves Test Factors}
The test evaluates:

\begin{enumerate}[label=\arabic*)]
 \item \textbf{Motivation of the Parties},
 \item \textbf{Plan of Distribution},
 \item \textbf{Reasonable Expectations of the Investing Public},
 \item \textbf{Existence of Alternative Regulatory Regimes}.
\end{enumerate}

\subsubsection{Motivation of the Parties}
\textbf{Analysis}:

- \textbf{Issuer's Motivation}: To facilitate the creation of a decentralized ecosystem where participants can access services and contribute value through their own efforts.
- \textbf{Participant's Motivation}: To use the tokens for immediate utility within the ecosystem and to engage in value-creating activities.
- \textbf{Embedded Disclaimers}: The smart contract's legal language informs participants of the token's purpose and non-investment nature.
- \textbf{Documentation and Agreements}: Token purchase agreements and terms of use emphasize access to services and participation, not investment.

\textbf{Conclusion}:

The motivations align with consumption and participation, not investment, indicating that the note is not a security.

\subsubsection{Plan of Distribution}
\textbf{Analysis}:

- \textbf{Targeted Distribution}: Tokens are distributed to individuals who intend to use them within the ecosystem, not marketed to investors seeking profits.
- \textbf{Distribution Channels}: Tokens are available through platforms that emphasize utility and participation, not through investment channels or securities exchanges.
- \textbf{Embedded Disclaimers}: Legal language within the smart contract is accessible to all purchasers, reinforcing the utility focus.
- \textbf{Limitations on Transferability}: The tokens may include features that limit their transferability, reinforcing their utility purpose.

\textbf{Conclusion}:

The distribution plan supports classification as a non-security instrument, as it is not broadly offered to potential investors but to users informed by embedded disclosures.

\subsubsection{Reasonable Expectations of the Investing Public}
\textbf{Analysis}:

- \textbf{Communications and Marketing}: All materials avoid investment language, focusing instead on utility, community engagement, and participant empowerment.
- \textbf{Embedded Disclaimers}: The smart contract provides clear statements about the token's non-security status, accessible to all participants.
- \textbf{Disclaimers}: Clear disclaimers state that tokens are not investment products and any value depends on participants' own efforts.
- \textbf{Participant Understanding}: Surveys and participant feedback indicate an understanding that tokens are for utility and personal contribution, not investment.

\textbf{Conclusion}:

The public does not have a reasonable expectation that tokens are securities, supporting the position that they are not notes under Reves. The embedded legal language in the smart contract reinforces this understanding.

\subsubsection{Existence of Alternative Regulatory Regimes}
\textbf{Analysis}:

- \textbf{Consumer Protection Laws}: Participants are protected under consumer protection and contractual laws.
- \textbf{Commodity Regulations}: To the extent applicable, tokens may be subject to commodity regulations, addressing potential fraud and market manipulation.
- \textbf{Embedded Disclosures}: The smart contract's legal disclaimers provide transparency and regulatory information, enhancing participant protection.
- \textbf{Self-Regulatory Measures}: Adoption of industry best practices, including transparency and participant protection initiatives.

\textbf{Regulatory Guidance}:

- The SEC's Strategic Hub for Innovation and Financial Technology (FinHub) provides frameworks and guidance that, when adhered to, offer alternative regulatory oversight.

\textbf{Conclusion}:

Alternative regulatory regimes provide sufficient oversight, reducing the need for securities regulation under the Reves Test.

\subsubsection{Definitive Conclusion Under Reves}
Considering the factors, CFSs, as structured—with embedded legal disclaimers—do not resemble notes that are securities. They function as consumer-oriented instruments facilitating access to content and enabling participant-driven value creation, not as investment vehicles.

\subsection{Addressing Potential Counterarguments and Enhancing the Argument}
To ensure the argument withstands the highest level of judicial scrutiny, we address potential concerns and substantiate the position with additional evidence.

\subsubsection{Effectiveness of Embedded Legal Disclaimers}
\textbf{Challenge}:

There may be skepticism regarding the effectiveness of disclaimers embedded within smart contracts, especially if purchasers do not read them or if they are not sufficiently prominent.

\textbf{Response and Mitigation}:

\begin{itemize}
\item \textbf{Accessibility}: The disclaimers are accessible to anyone interacting with the smart contract, and tools are available to view and verify the contract code.
\item \textbf{Emphasizing Visibility}: Accompanying user interfaces and platforms can highlight the presence of these disclaimers, encouraging participants to review them.
\item \textbf{Legal Precedent}: Courts have recognized disclosures in transactional documents as effective in clarifying the nature of an arrangement (\emph{See United Housing Foundation, Inc. v. Forman}).
\end{itemize}

\subsubsection{Participant Expectations and Market Behavior}
\textbf{Challenge}:

There is a concern that participants may acquire tokens with an expectation of profits due to secondary market trading and speculative behavior.

\textbf{Response and Mitigation}:

\begin{itemize}
\item \textbf{Issuer's Responsibilities}: The issuer does not promote secondary market trading and explicitly discourages speculative behavior.
\item \textbf{Embedded Disclaimers}: The smart contract's legal language informs participants that the token is not intended for investment purposes.
\item \textbf{Market Conduct Monitoring}: The issuer monitors market conduct to identify and address speculative activities inconsistent with the tokens' utility purpose.
\item \textbf{Participant Education}: Educational initiatives inform participants of the tokens' intended use and the risks associated with speculation.
\item \textbf{Empirical Data}: Statistical analysis shows the majority of token usage involves accessing services and participating in ecosystem activities rather than trading for profit.
\end{itemize}

\subsubsection{Control and Centralization Concerns}
\textbf{Challenge}:

There may be an argument that the issuer retains control, making participants reliant on their efforts, which could suggest an investment contract.

\textbf{Response and Mitigation}:

\begin{itemize}
\item \textbf{Decentralized Governance Structures}: Implementation of DAOs or similar mechanisms where participants have decision-making authority ensures that control is not centralized.
\item \textbf{Embedded Disclaimers}: Legal language in the smart contract emphasizes the decentralized nature and the limited role of the issuer.
\item \textbf{Transparency Reports}: Regular reports detailing the issuer's minimal role and the extensive contributions from participants enhance transparency.
\item \textbf{Third-Party Audits}: Independent audits confirm the decentralized nature of the ecosystem and the limited issuer control.
\end{itemize}

\subsubsection{Regulatory Guidance Alignment}
\textbf{Alignment with SEC Framework}:

\begin{itemize}
\item \textbf{Decentralized Network}: The network is fully operational and decentralized, with participants able to use tokens for their intended functionality immediately upon receipt.
\item \textbf{Absence of Reliance on a Central Entity}: The ecosystem does not rely on a central promoter to facilitate its ongoing operations or success.
\item \textbf{No Explicit or Implicit Promises of Profit}: Marketing and communications avoid any suggestions of potential profits, consistent with the SEC's guidance.
\item \textbf{Embedded Disclosures}: The smart contract's legal language aligns with regulatory expectations for transparency and participant information.
\end{itemize}

\textbf{Supporting No-Action Letters}:

\begin{itemize}
\item \textbf{Comparison to \emph{TurnKey Jet, Inc.} and \emph{Pocketful of Quarters, Inc.}}: In these cases, the SEC did not consider the tokens to be securities due to their functional utility and lack of investment features. The inclusion of embedded disclaimers in CFSs strengthens their alignment with these precedents.
\end{itemize}

\subsubsection{Recent Case Law Considerations}
\textbf{Case of \emph{SEC v. Ripple Labs Inc.} (2023)}:

\begin{itemize}
\item The court's decision highlighted the importance of context in determining whether a digital asset constitutes a security.
\item \textbf{Distinguishing Factors}: Unlike Ripple's XRP tokens, which were marketed and sold to institutional investors with expectations of profit from Ripple's efforts, CFSs are distributed to participants for immediate utility without promises of profit.
\item \textbf{Embedded Disclaimers}: The presence of legal language within the smart contract further distinguishes CFSs by providing clear disclosures to purchasers.
\end{itemize}

\textbf{Implications}:

The Ripple case underscores the necessity of examining the totality of circumstances. CFSs, as structured with embedded legal disclaimers, lack the features that led to XRP being considered an investment contract in certain contexts.

\subsubsection{Commodity Exchange Act Considerations}
Under the Commodity Exchange Act (CEA), certain digital assets may be considered commodities. However, the CEA primarily regulates derivatives, not spot transactions.

\textbf{Analysis}:

\begin{itemize}
\item \textbf{Consumptive Use}: Tokens are used for immediate access to services, not as commodities for speculative trading.
\item \textbf{Embedded Disclosures}: The smart contract informs participants of the token's intended utility use, aligning with non-commodity characteristics.
\item \textbf{Compliance with CEA}: To the extent applicable, the issuer ensures compliance with relevant CEA provisions, including anti-fraud and anti-manipulation regulations.
\end{itemize}

\textbf{Conclusion}:

CFSs function primarily as utility tokens and are not subject to extensive regulation under the CEA.

\section{Definitive Recommendations for Memecoin Developers}
To reinforce the argument and ensure compliance, memecoin developers should implement the following strategies:

\begin{enumerate}[label=\arabic*)]
 \item \textbf{Embed Legal Disclaimers in Smart Contracts}: Incorporate clear legal language directly into the smart contract code, providing immediate access to regulatory information for all purchasers.
 \item \textbf{Enhance Token Utility and Participant Empowerment}: Continuously develop and promote token functionalities that provide immediate utility and enable participants to create value through their own efforts.
 \item \textbf{Implement Decentralized Governance Models}: Establish governance mechanisms that allow participants to have meaningful control over ecosystem decisions, reducing reliance on a central entity.
 \item \textbf{Monitor and Control Communications}: Ensure all communications emphasize utility and participation, avoiding any language that could be interpreted as promoting investment opportunities.
 \item \textbf{Provide Clear Documentation and Disclaimers}: Offer comprehensive terms of use, disclaimers, and educational materials that clearly state the tokens' purpose and disavow any profit expectations, supplementing the embedded legal language.
 \item \textbf{Promote Accessibility of Disclaimers}: Ensure that the embedded legal language and any accompanying disclaimers are easily accessible and prominently displayed on user interfaces and platforms.
 \item \textbf{Engage in Participant Education}: Conduct workshops, webinars, and provide resources to educate participants on how to utilize tokens effectively and contribute to the ecosystem.
 \item \textbf{Conduct Independent Audits and Assessments}: Engage third-party auditors to verify the decentralized nature of the ecosystem, the limited role of the issuer, and the effectiveness of embedded disclaimers.
 \item \textbf{Adopt Self-Regulatory Measures and Best Practices}: Participate in industry associations, adhere to codes of conduct, and implement measures that enhance transparency and participant protection.
 \item \textbf{Maintain Open Regulatory Dialogue}: Proactively communicate with regulatory bodies, seeking guidance and feedback to ensure ongoing compliance.
 \item \textbf{Monitor Market Conduct}: Implement systems to monitor and address speculative trading activities that may conflict with the tokens' utility purpose.
 \item \textbf{Document Participant Contributions}: Collect and present evidence of participant-driven projects and initiatives that demonstrate the ecosystem's reliance on decentralized contributions.
\end{enumerate}

\section{Conclusion}
Through a comprehensive legal analysis and the implementation of robust structural features—including the embedding of legal disclaimers within smart contracts—we have demonstrated that Crowdfunding Swaps (CFSs), when properly structured, are not securities under U.S. federal securities laws. By meticulously addressing potential counterarguments with empirical evidence, enhancing decentralization, emphasizing participant-driven value creation, and providing transparent disclosures, memecoin developers can structure CFSs to fall outside the definition of securities.

Adhering to the recommended strategies will strengthen the legal position of memecoin developers, ensuring compliance and fostering innovation within the legal framework. By focusing on immediate utility, participant empowerment, incorporating embedded legal disclaimers, and minimizing issuer control, memecoin developers can mitigate regulatory risks and contribute positively to the development of decentralized ecosystems.

\begin{thebibliography}{99}
\bibitem{howey} \emph{SEC v. W.J. Howey Co.}, 328 U.S. 293 (1946).
\bibitem{unitedhousing} \emph{United Housing Foundation, Inc. v. Forman}, 421 U.S. 837 (1975).
\bibitem{reves} \emph{Reves v. Ernst \& Young}, 494 U.S. 56 (1990).
\bibitem{revak} \emph{Revak v. SEC Realty Corp.}, 18 F.3d 81 (2d Cir. 1994).
\bibitem{noa} \emph{Noa v. Key Futures, Inc.}, 638 F.2d 77 (9th Cir. 1980).
\bibitem{turnkey} Securities and Exchange Commission, \emph{TurnKey Jet, Inc. No-Action Letter}, April 3, 2019.
\bibitem{pocketful} Securities and Exchange Commission, \emph{Pocketful of Quarters, Inc. No-Action Letter}, July 25, 2019.
\bibitem{secframework} Securities and Exchange Commission, \emph{Framework for "Investment Contract" Analysis of Digital Assets}, April 2019.
\bibitem{ripple} \emph{SEC v. Ripple Labs Inc.}, No. 20-cv-10832 (S.D.N.Y. 2023).
\bibitem{lifepartners} \emph{SEC v. Life Partners, Inc.}, 87 F.3d 536 (D.C. Cir. 1996).
\bibitem{salman} \emph{Salman v. United States}, 137 S. Ct. 420 (2016).
\end{thebibliography}

\end{document}
